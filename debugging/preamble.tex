% this file was copied from my PhD thesis proposal project 2021-06-16
% It has manuy useful options and packages; many are probably not required for a 
% simple manuscript so I will clean up the extraneous ones that get in 
% the way but leave the others in case they prove useful 



% a useful website for knowing what things are called in latex
% http://www.texfaq.org/FAQ-fixnam



% packages I had already put in here 
\usepackage[none]{hyphenat}
\usepackage[font=small,labelfont=bf]{caption}


% all packages that kableExtra depends on 
\usepackage{booktabs}
\usepackage{longtable}
\usepackage{array}
\usepackage{multirow}
\usepackage{wrapfig}
\usepackage{float}
\usepackage{colortbl}
\usepackage{pdflscape}
\usepackage{tabu}
\usepackage{threeparttable}
\usepackage{threeparttablex}
\usepackage[normalem]{ulem}
% this one not needed because I always use UTF-8 \usepackage[utf8]{inputenc}
\usepackage{makecell}
\usepackage{xcolor}

%%%%%%%%%%%%%%%%%%%%%%%%%%%%%%%%%%%%%%%%%%%%%%%%%%%%%%%%%%%%%
% load cancel package for clearer unit conversions
\usepackage{cancel}
%%%%%%%%%%%%%%%%%%%%%%%%%%%%%%%%%%%%%%%%%%%%%%%%%%%%%%%%%%%%%

% apparently this package makes better bookmarks, 
% I need it for putting the TOC into the document tree
% see https://tex.stackexchange.com/questions/65544/how-to-link-table-of-contents-in-thesis-pdf

\usepackage{bookmark}



%%%%%%%%%%%%%%%%%%%%%%%%%%%%%%%%%%%%%%%%%%%%%%%%%%%%%%%%%%%%%


% set vertical line spacing for amsmath equations 
\setlength{\jot}{8pt}

% everything abve works well

% extras that I am testing come below 


% set paragraph indentation back to > 0; default is 20pt
\parindent=10pt



% define navy as a link color and use it for internal links
% depends on xcolor which is already loaded above 
\definecolor{dodgerblue4}{HTML}{104E8B}
\usepackage{hyperref}

\hypersetup{
  colorlinks=true,
  urlcolor=blue,
  linkcolor=dodgerblue4
}


% number the lines with small grey text, see https://tex.stackexchange.com/questions/247165/can-i-change-linenumber-colour
\usepackage{lineno}

% define R's grey75 color in HTML hex notation
\definecolor{grey75}{HTML}{BFBFBF}

% set color of line numbers 
% this was the code I copied from the URL above \renewcommand{\linenumberfont}{\normalfont\bfseries\small\color{grey75}}

% and this is my own version which uses the mono font. 
\renewcommand{\linenumberfont}{\ttfamily\small\color{grey75}}




%%%%%%%%%%%%%%%%%%%%%%%%%%%%%%%%%%%%


% re-style block quotations??
% there are 2 environments that handle block quotations- the quotation environment 
% is for multi-paragraph quotes and each paragraph is indented.
% the quote environment is meant for shorter quotes. Evidently R Markdown 
% puts at least the short ones in the quote environment; I discovered this by 
% inspecting the .tex file usingkeep_tex: true in the YAML


% see https://stackoverflow.com/questions/4018493/vertical-line-with-every-quotation/4023967
% for solution
\usepackage{framed}


% define R's grey30 color in HTML hex notation
\definecolor{grey30}{HTML}{4D4D4D}

% change bar size to 0.5 pt instead of the default 3 pt
% use the newly defined grey color 
\renewenvironment{leftbar}{\def\FrameCommand{\color{grey30}\vrule width 1pt \hspace{10pt}}\MakeFramed {\advance\hsize-\width \FrameRestore}}{\endMakeFramed}

% for multi-paragraph quotations 
% \renewenvironment{quotation}%
% {\begin{leftbar}\begin{quotation}}%
% {\end{quotation}\end{leftbar}}

% for single-paragraph quotations 
% remove indent with \noindent


\renewenvironment{quote}%
{\begin{leftbar} \begin{quotation} \noindent \small }%
{\end{quotation}\end{leftbar}}


