% Options for packages loaded elsewhere
\PassOptionsToPackage{unicode}{hyperref}
\PassOptionsToPackage{hyphens}{url}
\PassOptionsToPackage{dvipsnames,svgnames,x11names}{xcolor}
%
\documentclass[
  letterpaper,
]{article}
\usepackage{amsmath,amssymb}
\usepackage{lmodern}
\usepackage{setspace}
\usepackage{iftex}
\ifPDFTeX
  \usepackage[T1]{fontenc}
  \usepackage[utf8]{inputenc}
  \usepackage{textcomp} % provide euro and other symbols
\else % if luatex or xetex
  \usepackage{unicode-math}
  \defaultfontfeatures{Scale=MatchLowercase}
  \defaultfontfeatures[\rmfamily]{Ligatures=TeX,Scale=1}
  \setmainfont[]{Roboto}
  \setmathfont[]{Fira Math Regular}
\fi
% Use upquote if available, for straight quotes in verbatim environments
\IfFileExists{upquote.sty}{\usepackage{upquote}}{}
\IfFileExists{microtype.sty}{% use microtype if available
  \usepackage[]{microtype}
  \UseMicrotypeSet[protrusion]{basicmath} % disable protrusion for tt fonts
}{}
\makeatletter
\@ifundefined{KOMAClassName}{% if non-KOMA class
  \IfFileExists{parskip.sty}{%
    \usepackage{parskip}
  }{% else
    \setlength{\parindent}{0pt}
    \setlength{\parskip}{6pt plus 2pt minus 1pt}}
}{% if KOMA class
  \KOMAoptions{parskip=half}}
\makeatother
\usepackage{xcolor}
\usepackage[margin=1in]{geometry}
\usepackage{longtable,booktabs,array}
\usepackage{calc} % for calculating minipage widths
% Correct order of tables after \paragraph or \subparagraph
\usepackage{etoolbox}
\makeatletter
\patchcmd\longtable{\par}{\if@noskipsec\mbox{}\fi\par}{}{}
\makeatother
% Allow footnotes in longtable head/foot
\IfFileExists{footnotehyper.sty}{\usepackage{footnotehyper}}{\usepackage{footnote}}
\makesavenoteenv{longtable}
\usepackage{graphicx}
\makeatletter
\def\maxwidth{\ifdim\Gin@nat@width>\linewidth\linewidth\else\Gin@nat@width\fi}
\def\maxheight{\ifdim\Gin@nat@height>\textheight\textheight\else\Gin@nat@height\fi}
\makeatother
% Scale images if necessary, so that they will not overflow the page
% margins by default, and it is still possible to overwrite the defaults
% using explicit options in \includegraphics[width, height, ...]{}
\setkeys{Gin}{width=\maxwidth,height=\maxheight,keepaspectratio}
% Set default figure placement to htbp
\makeatletter
\def\fps@figure{htbp}
\makeatother
\setlength{\emergencystretch}{3em} % prevent overfull lines
\providecommand{\tightlist}{%
  \setlength{\itemsep}{0pt}\setlength{\parskip}{0pt}}
\setcounter{secnumdepth}{-\maxdimen} % remove section numbering
\newlength{\cslhangindent}
\setlength{\cslhangindent}{1.5em}
\newlength{\csllabelwidth}
\setlength{\csllabelwidth}{3em}
\newlength{\cslentryspacingunit} % times entry-spacing
\setlength{\cslentryspacingunit}{\parskip}
\newenvironment{CSLReferences}[2] % #1 hanging-ident, #2 entry spacing
 {% don't indent paragraphs
  \setlength{\parindent}{0pt}
  % turn on hanging indent if param 1 is 1
  \ifodd #1
  \let\oldpar\par
  \def\par{\hangindent=\cslhangindent\oldpar}
  \fi
  % set entry spacing
  \setlength{\parskip}{#2\cslentryspacingunit}
 }%
 {}
\usepackage{calc}
\newcommand{\CSLBlock}[1]{#1\hfill\break}
\newcommand{\CSLLeftMargin}[1]{\parbox[t]{\csllabelwidth}{#1}}
\newcommand{\CSLRightInline}[1]{\parbox[t]{\linewidth - \csllabelwidth}{#1}\break}
\newcommand{\CSLIndent}[1]{\hspace{\cslhangindent}#1}
% this file was originally copied from my PhD thesis proposal project 2021-06-16
% It has manuy useful options and packages; many are probably not required for a 
% simple manuscript so I will clean up the extraneous ones that get in 
% the way but leave the others in case they prove useful 



% a useful website for knowing what things are called in latex
% http://www.texfaq.org/FAQ-fixnam



% packages I had already put in here 
\usepackage[none]{hyphenat}
\usepackage[font=small,labelfont=bf]{caption}


% all packages that kableExtra depends on 
\usepackage{booktabs}
\usepackage{longtable}
\usepackage{array}
\usepackage{multirow}
\usepackage{wrapfig}
\usepackage{float}
\usepackage{colortbl}
\usepackage{pdflscape}
\usepackage{tabu}
\usepackage{threeparttable}
\usepackage{threeparttablex}
\usepackage[normalem]{ulem}
% this one not needed because I always use UTF-8 \usepackage[utf8]{inputenc}
\usepackage{makecell}

% allow for use of more colors 
% \usepackage[svgnames]{xcolor}
% \usepackage[]{xcolor}
\usepackage{xcolor}
\definecolor{psunavy}{HTML}{041E42}

% define a dark blue color for printing my comments 


%%%%%%%%%%%%%%%%%%%%%%%%%%%%%%%%%%%%%%%%%%%%%%%%%%%%%%%%%%%%%
% load cancel package for clearer unit conversions
\usepackage{cancel}
%%%%%%%%%%%%%%%%%%%%%%%%%%%%%%%%%%%%%%%%%%%%%%%%%%%%%%%%%%%%%

% apparently this package makes better bookmarks, 
% I need it for putting the TOC into the document tree
% see https://tex.stackexchange.com/questions/65544/how-to-link-table-of-contents-in-thesis-pdf

\usepackage{bookmark}



%%%%%%%%%%%%%%%%%%%%%%%%%%%%%%%%%%%%%%%%%%%%%%%%%%%%%%%%%%%%%


% set vertical line spacing for amsmath equations 
\setlength{\jot}{8pt}

% everything abve works well

% extras that I am testing come below 


% set paragraph indentation back to > 0; default is 20pt
\parindent=10pt



% define navy as a link color and use it for internal links
% depends on xcolor which is already loaded above 
\definecolor{dodgerblue4}{HTML}{104E8B}
\usepackage{hyperref}

\hypersetup{
  colorlinks=true,
  urlcolor=blue,
  linkcolor=dodgerblue4
}


% number the lines with small grey text, see https://tex.stackexchange.com/questions/247165/can-i-change-linenumber-colour
\usepackage{lineno}

% define R's grey75 color in HTML hex notation
\definecolor{grey75}{HTML}{BFBFBF}

% set color of line numbers 
% this was the code I copied from the URL above 
\renewcommand{\linenumberfont}{\normalfont\bfseries\small\color{grey75}}

% and this is my own version which uses the mono font. 
% \renewcommand{\linenumberfont}{\ttfamily\small\color{grey75}}




%%%%%%%%%%%%%%%%%%%%%%%%%%%%%%%%%%%%


% re-style block quotations??
% there are 2 environments that handle block quotations- the quotation environment 
% is for multi-paragraph quotes and each paragraph is indented.
% the quote environment is meant for shorter quotes. Evidently R Markdown 
% puts at least the short ones in the quote environment; I discovered this by 
% inspecting the .tex file usingkeep_tex: true in the YAML


% see https://stackoverflow.com/questions/4018493/vertical-line-with-every-quotation/4023967
% for solution
\usepackage{framed}


% define R's grey30 color in HTML hex notation
\definecolor{grey30}{HTML}{4D4D4D}

% change bar size to 0.5 pt instead of the default 3 pt
% use the newly defined grey color 
\renewenvironment{leftbar}{\def\FrameCommand{\color{grey30}\vrule width 1pt \hspace{10pt}}\MakeFramed {\advance\hsize-\width \FrameRestore}}{\endMakeFramed}

% for multi-paragraph quotations 
% \renewenvironment{quotation}%
% {\begin{leftbar}\begin{quotation}}%
% {\end{quotation}\end{leftbar}}

% for single-paragraph quotations 
% remove indent with \noindent


\renewenvironment{quote}%
{\begin{leftbar} \begin{quotation} \noindent \small }%
{\end{quotation}\end{leftbar}}


% to remove parentheses around equations 
% copied from https://tex.stackexchange.com/questions/318627/remove-parentheses-that-surround-equation-labels

\usepackage{mathtools}
% not needed since this package is already loaded \usepackage{hyperref}
\usepackage[capitalise,nameinlink,noabbrev]{cleveref}

\creflabelformat{equation}{#2#1#3}

\newtagform{noparen}{}{}
\usetagform{noparen}

\usepackage{booktabs}
\usepackage{longtable}
\usepackage{array}
\usepackage{multirow}
\usepackage{wrapfig}
\usepackage{float}
\usepackage{colortbl}
\usepackage{pdflscape}
\usepackage{tabu}
\usepackage{threeparttable}
\usepackage{threeparttablex}
\usepackage[normalem]{ulem}
\usepackage{makecell}
\usepackage{xcolor}
\usepackage{fontspec}
\usepackage{multicol}
\usepackage{hhline}
\usepackage{hyperref}
\ifLuaTeX
  \usepackage{selnolig}  % disable illegal ligatures
\fi
\IfFileExists{bookmark.sty}{\usepackage{bookmark}}{\usepackage{hyperref}}
\IfFileExists{xurl.sty}{\usepackage{xurl}}{} % add URL line breaks if available
\urlstyle{same} % disable monospaced font for URLs
\hypersetup{
  pdftitle={Coarse additions affect the plasticity and toughness of soil mixtures, Part II: Sand angularity and sand-size uniformity},
  colorlinks=true,
  linkcolor={blue},
  filecolor={Maroon},
  citecolor={Blue},
  urlcolor={blue},
  pdfcreator={LaTeX via pandoc}}

\title{Coarse additions affect the plasticity and toughness of soil mixtures, Part II: Sand angularity and sand-size uniformity}
\author{}
\date{\vspace{-2.5em}last compiled Fri. 2022-08-12, 12:48 PM}

\begin{document}
\maketitle

{
\hypersetup{linkcolor=blue}
\setcounter{tocdepth}{2}
\tableofcontents
}
\setstretch{1.5}
\linenumbers

\hypertarget{abstract}{%
\section{Abstract}\label{abstract}}

The Atterberg limits may offer a useful means to evaluate baseball infield soils because they quantitatively relate soil behavior to water content.
Prior research has demonstrated that liquid and plastic limits (LL and PL) of sand-clay mixtures are affected by the quantity and type of admixed sand, but these studies have used \textless425 μm sand exclusively and little attention has been devoted to sand angularity and sand-size uniformity.

This research was conducted to clarify the effect of sand angularity and sand-size uniformity on the Atterberg limits of soil mixtures containing a range of sand contents and a significant mass percentage 425-2000 μm.

Experiment 1 compared the effect of mixing either an angular or a round sand (both 0.5-1 mm) with a kaolinitic clay at sand contents between 0 and 80\%.
Little difference was observed in LL and PL, suggesting angularity plays a minimal role on mix performance.

Experiment 2 compared the effect of mixing one of two sands having similar D\textsubscript{50} (0.42 and 0.49 mm) but varying uniformity (uniformity coefficients of 1.9 vs.~3.9) with an illitic clay at sand content 0-80\%.
Mixtures including the high-C\textsubscript{u} sand maintained their plasticity to higher sand content (\textasciitilde72.5\%) than those produced with the low-C\textsubscript{u} sand (\textasciitilde67.5\%).

Calculations for the threshold fines content and intergranular porosity agreed closely with the experiments, indicating a potential to estimate TFC from sand porosity alone.

\hypertarget{introduction}{%
\section{Introduction}\label{introduction}}

The Atterberg limits are used by geotechnical engineers to classify fine-grained soils (\protect\hyperlink{ref-ASTMInternational2017}{ASTM International, 2017}) and predict their behavior upon loading.
These tests include the liquid limit (LL) and plastic limit (PL).
The tests were developed for fine-grained soils, but research has shown the amount and properties of admixed sand also affects the test results (\protect\hyperlink{ref-Atterberg1911}{Atterberg, 1911}; \protect\hyperlink{ref-Dumbleton1966b}{Dumbleton and West, 1966}; \protect\hyperlink{ref-Sivapullaiah1985}{Sivapullaiah and Sridharan, 1985}; \protect\hyperlink{ref-Barnes2013}{Barnes, 2013}).
Particle angularity and particle-size uniformity are known to affect the behavior of granular materials, but their effects are less clear for soils which contain significant amounts of fines (\protect\hyperlink{ref-Mitchell1993}{Mitchell and Soga, 1993}; \protect\hyperlink{ref-Mittal2004}{Mittal et al., 2004}; \protect\hyperlink{ref-Holtz2010}{Holtz et al., 2010}; \protect\hyperlink{ref-Miller2011}{Miller and Henderson, 2011}; \protect\hyperlink{ref-Zuo2015}{Zuo and Baudet, 2015}).

\hypertarget{atterberg-limits-of-soils-having-varied-sand-angularity}{%
\subsection{Atterberg limits of soils having varied sand angularity}\label{atterberg-limits-of-soils-having-varied-sand-angularity}}

Particle angularity is defined as the relative protrusion (roughness) of a particle surface (\protect\hyperlink{ref-Wadell1932}{Wadell, 1932}).
Mathematical definitions have been used to quantify angularity, although it is often estimated using representative charts developed from the metrics (\protect\hyperlink{ref-Wadell1932}{Wadell, 1932}; \protect\hyperlink{ref-Krumbein1941}{Krumbein, 1941}; \protect\hyperlink{ref-Suhr2020}{Suhr et al., 2020}).

Research on the importance of sand angularity in soil mixtures has produced mixed results.
Dumbleton and West (\protect\hyperlink{ref-Dumbleton1966b}{1966}) evaluated mixtures of various types of coarse particles with either kaolinite or montmorillonite.
This research demonstrated that mixtures including angular sand had higher LL and PL relative to mixtures produced with equivalent amounts of glass spheres.
Dumbleton and West (\protect\hyperlink{ref-Dumbleton1966b}{1966}) postulated that angular particles had more surface area compared to the spheres, requiring additional water to coat their surfaces before they could slide or flow past one another.

Findings by Sivapullaiah and Sridharan (\protect\hyperlink{ref-Sivapullaiah1985}{1985}) differ from those of Dumbleton and West (\protect\hyperlink{ref-Dumbleton1966b}{1966}).
Sivapullaiah and Sridharan (\protect\hyperlink{ref-Sivapullaiah1985}{1985}) mixed either angular or round particles of the same size fraction (150 μm -- 75 μm) with bentonite clay at sand contents ranging from 20-95 \%.
Sivapullaiah and Sridharan (\protect\hyperlink{ref-Sivapullaiah1985}{1985}) reported no difference in the liquid or plastic limits due to angularity.

\hypertarget{atterberg-limits-of-soils-having-varied-sand-uniformity}{%
\subsection{Atterberg limits of soils having varied sand uniformity}\label{atterberg-limits-of-soils-having-varied-sand-uniformity}}

Particle-size uniformity, or more simply ``uniformity,'' is defined as the similarity of diameters across particles within a soil.
The coefficient of uniformity (C\textsubscript{u}) is a commonly used mathematical definition (\protect\hyperlink{ref-Adams1994}{Adams and Gibbs, 1994}; \protect\hyperlink{ref-Holtz2010}{Holtz et al., 2010}).
C\textsubscript{u} is the ratio between the particle diameters at which 60\% and 10\% of the sample is finer:

\begin{equation}
C_u = \frac{D_{60}}{D_{10}}
\label{eq:Cu-equation}
\end{equation}

Little research is available on the effect of sand uniformity on the Atterberg limits of soil mixtures.
This is probably because the upper boundary of allowed particle sizes in ASTM D4318 (425 μm) limits the total range of sand particle diameters.
Dumbleton and West (\protect\hyperlink{ref-Dumbleton1966b}{1966}) reported that mixes produced using non-uniform sand had similar Atterberg limits to mixes containing sand from a single mesh size.
However, all the sand in their experiment was between 425 and 53 μm, limiting the potential variability in uniformity.
Efficient particle packing - defined as the ability of smaller particles to fit comfortably in the voids between larger particles - is limited when the majority of particle diameters fall within a factor of 10 (\protect\hyperlink{ref-Lade1998}{Lade et al., 1998}).
It is possible that a larger uniformity effect on LL and PL would be observed if coarser sand were included in the mixture to create a higher C\textsubscript{u}.

At very low sand contents, angularity and uniformity are probably unimportant because the sand grains are suspended in a clay-water matrix and do play an insignificant role in transmitting loads through the soil skeleton.
At higher sand contents, the coarse grains begin to contact one another and the nature of these contacts will affect the behavior of the bulk soil (\protect\hyperlink{ref-Mitchell1993}{Mitchell and Soga, 1993}; \protect\hyperlink{ref-Zuo2015}{Zuo and Baudet, 2015}).
As sand content continues to increase, the soil becomes nonplastic because plasticity is not observed in soils dominated by granular particles (\protect\hyperlink{ref-Mitchell1993}{Mitchell and Soga, 1993}; \protect\hyperlink{ref-Holtz2010}{Holtz et al., 2010}).
The behavior transition between sand-like and fines-like behavior occurs at a fines content termed the threshold fines content (TFC) (\protect\hyperlink{ref-Zuo2015}{Zuo and Baudet, 2015}; \protect\hyperlink{ref-Sibley2020}{Sibley and Polito, 2020}).
TFC may be computed as

\begin{equation}
m_{sa} = \frac{1 +  \frac{\frac{w G_c }{G_w}}{S_e}}{1 + \frac{e_{sa~(max)} G_c}{G_{sa}}}
\label{eq:tfc-equation}
\end{equation}

where \(m_{sa}\) is the sand mass, \(w\) is the gravimetric water content, \(S_e\) is the effective saturation, \(G_w\) is the specific gravity of water at 20 °C, \(G_{sa} and G_c\) are the specific gravities of the sand and clay, and \(e_{sa}\) is the void ratio of the sand.

Because less-uniform sands have less total void space, one could expect these sands to require a lesser proportion of fines to fill their interstices, compared with a more uniform sand.
Therefore, sand uniformity could conceivably alter the upper limit of sand content at which the soil will exhibit plasticity.

\hypertarget{objectives}{%
\section{Objectives}\label{objectives}}

Recent research has demonstrated that the LL and PL tests can be performed on soils containing a significant amount of particles 2000-425 μm (Mascitti and McNitt, this issue).
It is unclear what effect sand shape and uniformity play on the Atterberg limits of soil mixtures containing these coarser sands.

The present research was conducted to answer two questions:

\begin{enumerate}
\def\labelenumi{\arabic{enumi}.}
\item
  How does sand particle angularity affect LL and PL of sand-clay mixtures when total sand content and sand particle size are held constant?
\item
  How does sand-size uniformity affect LL and PL of sand-clay mixtures when total sand content and \emph{average} sand particle size (D\textsubscript{50}) are held constant?
\end{enumerate}

\hypertarget{materials-and-methods}{%
\section{Materials and methods}\label{materials-and-methods}}

Two experiments were conducted to evaluate the effects of sand angularity and uniformity on the Atterberg limits of soil mixtures containing a range of sand contents.

\hypertarget{mixture-component-characterization}{%
\subsection{Mixture component characterization}\label{mixture-component-characterization}}

Particle size analyses were performed on each of the four sands and the two clay components used in the two experiments.
For the fraction \textless53 μm, gravity sedimentation plus centrifugation were used along with the pipette method to compute the percent of the sample finer than 20, 5, 2, and 0.2 μm (\protect\hyperlink{ref-Gee2002}{Gee and Or, 2002}).
Particles \textgreater53 μm were separated using a Ro-tap shaker (W.S. Tyler, Mentor, OH) and a stack of mesh sieves.

The minimum and maximum void ratios of the sand components used in each experiment were characterized using modified versions of ASTM (\protect\hyperlink{ref-ASTMF1815-11}{2011}, \protect\hyperlink{ref-ASTMInternational2016}{2016}).
In the minimum void ratio test , a single lift of sand was dynamically compacted using a specified drop weight.
In the maximum void ratio test, the sand was carefully poured into a container of known volume and leveled with a straight edge.

The liquid and plastic limit of each clay component were determined individually using ASTM D4318 (\protect\hyperlink{ref-ASTMD43182018}{ASTM International, 2018}).
The LL of the kaolinitic clay used in Experiment 1 was 43 and its PL was 17, yielding a plasticity index of 26.
The PL of the illitic clay soil used in Experiment 2 was 18 and its LL was 30, yielding a PI of 12.
The particle size distribution of the illitic clay soil used in Experiment 2 is also shown in Figure \ref{fig:uniformity-experiment-particle-size-curves}.

\hypertarget{mixing-procedure}{%
\subsection{Mixing procedure}\label{mixing-procedure}}

All mixture components were air-dried and their water contents were determined gravimetrically.
The clay components were pulverized and passed though 0.25 mm screen.
Sands were mixed by hand with the relevant clay component until visually homogeneous.
The mixture component percentages were adjusted for the trace amounts of particles 2000-53 μm in the clay components.
Final mixture percentages are expressed as oven-dry mass.

\hypertarget{treatments}{%
\subsection{Treatments}\label{treatments}}

Experiment 1 evaluated mixtures containing equivalent amounts of one of two sands having different angularity.
Two sands were selected based on their classification as angular and well-rounded.
The sands were visually classified using a microscope and the chart from Baker (\protect\hyperlink{ref-Baker2006a}{2006}) (Figure \ref{fig:sand-photomicrographs}).
Each sand was then repeatedly sieved to remove particles \textless0.5 mm and \textgreater1 mm.
The single-mesh sands were mixed with a kaolinitic clay to yield mixtures having 0, 20, 40, 60, 70, 75, and 80 \% sand.

\begin{figure}

{\centering \includegraphics[width=0.9\linewidth]{E:/OneDrive - The Pennsylvania State University/PSU2019-present/A_inf_soils_PhD/oversizeALims/figs/png/image-figs/sand-photo-micrographs} 

}

\caption{Angular (A.) and well-rounded (B.) sands used for Experiment 1.}\label{fig:sand-photomicrographs}
\end{figure}

Experiment 2 evaluated the effect of sand-size uniformity for mixtures containing one of two sands having similar D\textsubscript{50} but varying C\textsubscript{u}.
The two sands were produced from a single washed concrete sand having a wide particle size distribution.
The original concrete sand was sieved to remove all particles \textgreater1 mm and \textless0.25 mm.
The remaining fraction between 1 and 0.5 mm was separated into two aliquots using a riffle splitter.
The first aliquot remained untouched and is termed the ``low-C\textsubscript{u} sand''.
The second aliquot, termed the ``high-C\textsubscript{u} sand'', had a portion of the removed particles (\textgreater1 mm and \textless0.25 mm) returned.
This procedure created a wider particle-size distribution while maintaining a similar D\textsubscript{50}.
Figure \ref{fig:uniformity-experiment-particle-size-curves} shows the particle size distributions of the low-C\textsubscript{u} and high-C\textsubscript{u} sands.
These sands meet the criteria of having similar D\textsubscript{50} values (0.42 mm and 0.49 mm) but different C\textsubscript{u} values (1.9 vs.~3.9).
While the C\textsubscript{u} value of 3.9 is still relatively low compared to natural alluvial sands, in this experiment the maximum obtainable C\textsubscript{u} was limited by the maximum particle diameter of 2000 μm and the intentionally limited mass of particles \textless425 μm.

Each of these sands were mixed with a single illitic clay soil to yield mixtures having between 0 and 80\% sand, for a total of 42 mixtures.
Mixtures were produced at 5\% sand content intervals between 0 and 50\%, and a 2.5\% interval between 50 and 80\% sand.
The tighter spacing of data points between 50-80\% sand was designed to provide better resolution near the threshold fines content.

\begin{figure}

{\centering \includegraphics[width=0.9\linewidth]{E:/OneDrive - The Pennsylvania State University/PSU2019-present/A_inf_soils_PhD/oversizeALims/figs/pdf/uniformity-experiment-particle-size-curves} 

}

\caption{Particle size distributions for the two sands and one clay used in Experiment 2. Dashed grey lines indicate $D_{50}$ for each sand.}\label{fig:uniformity-experiment-particle-size-curves}
\end{figure}

\hypertarget{atterberg-limit-test-protocol}{%
\subsection{Atterberg limit test protocol}\label{atterberg-limit-test-protocol}}

After mixing the sand with the relevant clay component, LL and PL tests were performed on a series of mixtures to evaluate the effect of sand angularity or uniformity.

The liquid and plastic limit tests were performed according to a modified version of ASTM D4318 (\protect\hyperlink{ref-ASTMD43182018}{ASTM International, 2018}).
The modification eliminated the wet-sieving procedure so particles between 2000 and 425 μm remained in the sample (Mascitti and McNitt, this issue).
At least four data points were collected during the LL test in order to plot the flow curve.
In the PL test, 3 threads were rolled to the crumbling condition before being weighed to ± 0.001 g and oven-dried.
In both experiments the average of the 3 results was used to represent the PL of each sample.

\hypertarget{statistical-analysis-and-computational-environment}{%
\subsection{Statistical analysis and computational environment}\label{statistical-analysis-and-computational-environment}}

LL and PL were the dependent variables in both experiments.
A two-way ANOVA model was fitted to test the interaction effect between particle shape and percent sand (Table \ref{tab:shape-experiment-anova-table})
In Experiment 1, particle shape was considered a categorical predictor while a 2nd-order polynomial spline term was used to model percent sand as a continuous predictor.
In Experiment 2, C\textsubscript{u} was considered a categorical predictor while percent sand was treated as in Experiment 1.
Main effects and interactions were tested using Type III Sums of Squares.
Treatments were considered significantly different at α = 0.05.

All analyses were performed using the \texttt{lm()} function in the R Language for Statistical Computing (version 4.2.0) (\protect\hyperlink{ref-R-Core-Team2022}{R-Core-Team, 2022}).
GNU \texttt{Make} (\protect\hyperlink{ref-GNU2020}{GNU, 2020}) was used to facilitate reproducible analyses by maintaining links between raw data, analysis code, and finished output.
A \texttt{Makefile} which runs the relevant analysis code is included in the supplemental materials.

\hypertarget{results-and-discussion}{%
\section{Results and discussion}\label{results-and-discussion}}

\hypertarget{experiment-1-effect-of-sand-angularity}{%
\subsection{Experiment 1: Effect of sand angularity}\label{experiment-1-effect-of-sand-angularity}}

Experiment 1 compared the effect of angularity for two sands of equal size ranging from 0-80\% sand.
Figure \ref{fig:shape-experiment-atterberg-limit-facets} shows that the LL and PL were nearly identical for sand content \textless60\%.
At sand content ≥ 60\%, a very slight increase in both LL and PL is visible for the angular sand, but this increase was not statistically significant.

The maximum difference between angular and round sand for any of the tests was 1.0 \% water content.
Under the conditions of this study, particle angularity is not important when sand size and sand contents are held constant.

\begin{figure}

{\centering \includegraphics[width=0.9\linewidth]{E:/OneDrive - The Pennsylvania State University/PSU2019-present/A_inf_soils_PhD/oversizeALims/figs/pdf/experiment-2-atterberg-limit-facets} 

}

\caption{The effect of particle shape on both the LL and PL tests was minimal.}\label{fig:shape-experiment-atterberg-limit-facets}
\end{figure}

\begin{table}

\caption{\label{tab:shape-experiment-anova-table}Analysis of variance table for each characteristic water content in Experiment 1.}
\centering
\begin{tabular}[t]{llrrll}
\toprule
\textbf{Test type} & \textbf{Term} & \textbf{Sum Sq.} & \textbf{Deg. of Fr.} & \textbf{F-Statistic} & \textbf{P-value}\\
\midrule
\addlinespace[0.3em]
\multicolumn{6}{l}{\textbf{}}\\
\hline
 & Intercept & 0.2102 & 1 & 18851.7 & <0.001\\
\cmidrule{2-6}
 & \% coarse addition & 0.0663 & 2 & 2971.9 & <0.001\\
\cmidrule{2-6}
 & Shape & 0.0000 & 1 & 0.6 & 0.47\\
\cmidrule{2-6}
 & \% coarse addition x Shape & 0.0001 & 2 & 2.5 & 0.16\\
\cmidrule{2-6}
\multirow{-5}{*}{\raggedright\arraybackslash \hspace{1em}LL} & Residuals & 0.0001 & 6 & - & -\\
\cmidrule{1-6}
\addlinespace[0.3em]
\multicolumn{6}{l}{\textbf{}}\\
\hline
 & Intercept & 0.0336 & 1 & 7179.3 & <0.001\\
\cmidrule{2-6}
 & \% coarse addition & 0.0065 & 2 & 696.7 & <0.001\\
\cmidrule{2-6}
 & Shape & 0.0000 & 1 & 0.1 & 0.8\\
\cmidrule{2-6}
 & \% coarse addition x Shape & 0.0000 & 2 & 1.1 & 0.4\\
\cmidrule{2-6}
\multirow{-5}{*}{\raggedright\arraybackslash \hspace{1em}PL} & Residuals & 0.0000 & 4 & - & -\\
\bottomrule
\end{tabular}
\end{table}

\hypertarget{experiment-2-effect-of-sand-size-uniformity}{%
\subsection{Experiment 2: Effect of sand-size uniformity}\label{experiment-2-effect-of-sand-size-uniformity}}

\hypertarget{uniformity-effect-on-ll-and-pl}{%
\subsubsection{Uniformity effect on LL and PL}\label{uniformity-effect-on-ll-and-pl}}

The LL and PL of both mixtures including either sand were inversely proportional to sand content, up to the higher end of the range (\textasciitilde65-70\%\%), above which the relationship is less clear.
Both sands in Experiment 2 were relatively coarse (D\textsubscript{50} of 0.42 and 0.49 mm), and the strong inverse relationship agrees with other research on mixtures containing coarse sand (Mascitti and McNitt, Part I, this issue).

Mixtures produced with low-C\textsubscript{u} sand had higher LL and PL than those produced using high-C\textsubscript{u} sand (Figure \ref{fig:uniformity-experiment-atterberg-limit-facets}).
This small effect was statistically significant for the LL but not for the PL (Table \ref{tab:uniformity-experiment-anova-table}).
Although this effect is measurable, it is probably of little practical significance because the differences were almost always \textless1\% water content.
The sand content was more influential on LL and PL than was sand uniformity.
Also, while only a single clay was tested, the effect of varying clay plasticity would likely affect LL and PL to a greater degree than varying the C\textsubscript{u} of the sand.

Dumbleton and West (\protect\hyperlink{ref-Dumbleton1966b}{1966}) suggested that observed differences in LL and PL due to sand particle size were attributable to higher specific surface area (SSA).
The nonlinear relationship between SSA and particle diameter dictates means requires that the low-C\textsubscript{u} sand would consume more water and increase the water content of the clay matrix.

\begin{figure}

{\centering \includegraphics[width=0.9\linewidth]{E:/OneDrive - The Pennsylvania State University/PSU2019-present/A_inf_soils_PhD/oversizeALims/figs/pdf/uniformity-experiment-atterberg-limit-facets} 

}

\caption{The high-Cu sand had higher LL and PL than the low-Cu sand at nearly all sand contents.}\label{fig:uniformity-experiment-atterberg-limit-facets}
\end{figure}

\begin{table}

\caption{\label{tab:uniformity-experiment-anova-table}Analysis of variance table for each characteristic water content in Experiment 2. Significant effects at α=0.05 in bold.}
\centering
\begin{tabular}[t]{llrrll}
\toprule
\textbf{Test type} & \textbf{Term} & \textbf{Sum Sq.} & \textbf{Deg. of Fr.} & \textbf{F-Statistic} & \textbf{P-value}\\
\midrule
\addlinespace[0.3em]
\multicolumn{6}{l}{\textbf{}}\\
\hline
\textbf{} & \textbf{Intercept} & \textbf{0.4116} & \textbf{1} & \textbf{24715.2} & \textbf{<0.001}\\
\cmidrule{2-6}
\textbf{} & \textbf{\% sand} & \textbf{0.0659} & \textbf{1} & \textbf{3954} & \textbf{<0.001}\\
\cmidrule{2-6}
\textbf{} & \textbf{Uniformity} & \textbf{0.0001} & \textbf{1} & \textbf{6} & \textbf{0.019}\\
\cmidrule{2-6}
 & \% sand x Uniformity & 0.0000 & 1 & 0.1 & 0.81\\
\cmidrule{2-6}
\multirow{-5}{*}{\raggedright\arraybackslash \hspace{1em}LL} & Residuals & 0.0006 & 36 & - & -\\
\cmidrule{1-6}
\addlinespace[0.3em]
\multicolumn{6}{l}{\textbf{}}\\
\hline
\textbf{} & \textbf{Intercept} & \textbf{0.1315} & \textbf{1} & \textbf{8914.4} & \textbf{<0.001}\\
\cmidrule{2-6}
\textbf{} & \textbf{\% sand} & \textbf{0.0137} & \textbf{1} & \textbf{929} & \textbf{<0.001}\\
\cmidrule{2-6}
 & Uniformity & 0.0000 & 1 & 1.3 & 0.26\\
\cmidrule{2-6}
 & \% sand x Uniformity & 0.0000 & 1 & 1.6 & 0.21\\
\cmidrule{2-6}
\multirow{-5}{*}{\raggedright\arraybackslash \hspace{1em}PL} & Residuals & 0.0004 & 30 & - & -\\
\bottomrule
\end{tabular}
\end{table}

\providecommand{\docline}[3]{\noalign{\global\setlength{\arrayrulewidth}{#1}}\arrayrulecolor[HTML]{#2}\cline{#3}}

\setlength{\tabcolsep}{2pt}

\renewcommand*{\arraystretch}{1.5}

\begin{longtable}[c]{|p{0.75in}|p{0.75in}|p{0.75in}|p{0.75in}|p{0.75in}|p{0.75in}}

\caption{Analysis of variance table for each characteristic water content in Experiment 2. Significant effects at α=0.05 in bold.
}\label{tab:uniformity-experiment-anova-table}\\

\hhline{>{\arrayrulecolor[HTML]{666666}\global\arrayrulewidth=2pt}->{\arrayrulecolor[HTML]{666666}\global\arrayrulewidth=2pt}->{\arrayrulecolor[HTML]{666666}\global\arrayrulewidth=2pt}->{\arrayrulecolor[HTML]{666666}\global\arrayrulewidth=2pt}->{\arrayrulecolor[HTML]{666666}\global\arrayrulewidth=2pt}->{\arrayrulecolor[HTML]{666666}\global\arrayrulewidth=2pt}-}

\multicolumn{1}{!{\color[HTML]{000000}\vrule width 0pt}>{\raggedright}p{\dimexpr 0.75in+0\tabcolsep+0\arrayrulewidth}}{\fontsize{11}{11}\selectfont{\textcolor[HTML]{000000}{\global\setmainfont{Arial}{Test\ type}}}} & \multicolumn{1}{!{\color[HTML]{000000}\vrule width 0pt}>{\raggedright}p{\dimexpr 0.75in+0\tabcolsep+0\arrayrulewidth}}{\fontsize{11}{11}\selectfont{\textcolor[HTML]{000000}{\global\setmainfont{Arial}{Term}}}} & \multicolumn{1}{!{\color[HTML]{000000}\vrule width 0pt}>{\raggedleft}p{\dimexpr 0.75in+0\tabcolsep+0\arrayrulewidth}}{\fontsize{11}{11}\selectfont{\textcolor[HTML]{000000}{\global\setmainfont{Arial}{Sum\ Sq.}}}} & \multicolumn{1}{!{\color[HTML]{000000}\vrule width 0pt}>{\raggedleft}p{\dimexpr 0.75in+0\tabcolsep+0\arrayrulewidth}}{\fontsize{11}{11}\selectfont{\textcolor[HTML]{000000}{\global\setmainfont{Arial}{Deg.\ of\ Fr.}}}} & \multicolumn{1}{!{\color[HTML]{000000}\vrule width 0pt}>{\raggedleft}p{\dimexpr 0.75in+0\tabcolsep+0\arrayrulewidth}}{\fontsize{11}{11}\selectfont{\textcolor[HTML]{000000}{\global\setmainfont{Arial}{F-Statistic}}}} & \multicolumn{1}{!{\color[HTML]{000000}\vrule width 0pt}>{\raggedleft}p{\dimexpr 0.75in+0\tabcolsep+0\arrayrulewidth}!{\color[HTML]{000000}\vrule width 0pt}}{\fontsize{11}{11}\selectfont{\textcolor[HTML]{000000}{\global\setmainfont{Arial}{P-value}}}} \\

\hhline{>{\arrayrulecolor[HTML]{666666}\global\arrayrulewidth=2pt}->{\arrayrulecolor[HTML]{666666}\global\arrayrulewidth=2pt}->{\arrayrulecolor[HTML]{666666}\global\arrayrulewidth=2pt}->{\arrayrulecolor[HTML]{666666}\global\arrayrulewidth=2pt}->{\arrayrulecolor[HTML]{666666}\global\arrayrulewidth=2pt}->{\arrayrulecolor[HTML]{666666}\global\arrayrulewidth=2pt}-}

\endfirsthead

\hhline{>{\arrayrulecolor[HTML]{666666}\global\arrayrulewidth=2pt}->{\arrayrulecolor[HTML]{666666}\global\arrayrulewidth=2pt}->{\arrayrulecolor[HTML]{666666}\global\arrayrulewidth=2pt}->{\arrayrulecolor[HTML]{666666}\global\arrayrulewidth=2pt}->{\arrayrulecolor[HTML]{666666}\global\arrayrulewidth=2pt}->{\arrayrulecolor[HTML]{666666}\global\arrayrulewidth=2pt}-}

\multicolumn{1}{!{\color[HTML]{000000}\vrule width 0pt}>{\raggedright}p{\dimexpr 0.75in+0\tabcolsep+0\arrayrulewidth}}{\fontsize{11}{11}\selectfont{\textcolor[HTML]{000000}{\global\setmainfont{Arial}{Test\ type}}}} & \multicolumn{1}{!{\color[HTML]{000000}\vrule width 0pt}>{\raggedright}p{\dimexpr 0.75in+0\tabcolsep+0\arrayrulewidth}}{\fontsize{11}{11}\selectfont{\textcolor[HTML]{000000}{\global\setmainfont{Arial}{Term}}}} & \multicolumn{1}{!{\color[HTML]{000000}\vrule width 0pt}>{\raggedleft}p{\dimexpr 0.75in+0\tabcolsep+0\arrayrulewidth}}{\fontsize{11}{11}\selectfont{\textcolor[HTML]{000000}{\global\setmainfont{Arial}{Sum\ Sq.}}}} & \multicolumn{1}{!{\color[HTML]{000000}\vrule width 0pt}>{\raggedleft}p{\dimexpr 0.75in+0\tabcolsep+0\arrayrulewidth}}{\fontsize{11}{11}\selectfont{\textcolor[HTML]{000000}{\global\setmainfont{Arial}{Deg.\ of\ Fr.}}}} & \multicolumn{1}{!{\color[HTML]{000000}\vrule width 0pt}>{\raggedleft}p{\dimexpr 0.75in+0\tabcolsep+0\arrayrulewidth}}{\fontsize{11}{11}\selectfont{\textcolor[HTML]{000000}{\global\setmainfont{Arial}{F-Statistic}}}} & \multicolumn{1}{!{\color[HTML]{000000}\vrule width 0pt}>{\raggedleft}p{\dimexpr 0.75in+0\tabcolsep+0\arrayrulewidth}!{\color[HTML]{000000}\vrule width 0pt}}{\fontsize{11}{11}\selectfont{\textcolor[HTML]{000000}{\global\setmainfont{Arial}{P-value}}}} \\

\hhline{>{\arrayrulecolor[HTML]{666666}\global\arrayrulewidth=2pt}->{\arrayrulecolor[HTML]{666666}\global\arrayrulewidth=2pt}->{\arrayrulecolor[HTML]{666666}\global\arrayrulewidth=2pt}->{\arrayrulecolor[HTML]{666666}\global\arrayrulewidth=2pt}->{\arrayrulecolor[HTML]{666666}\global\arrayrulewidth=2pt}->{\arrayrulecolor[HTML]{666666}\global\arrayrulewidth=2pt}-}\endhead



\multicolumn{1}{!{\color[HTML]{000000}\vrule width 0pt}>{\raggedright}p{\dimexpr 0.75in+0\tabcolsep+0\arrayrulewidth}}{\fontsize{11}{11}\selectfont{\textcolor[HTML]{000000}{\global\setmainfont{Arial}{LL}}}} & \multicolumn{1}{!{\color[HTML]{000000}\vrule width 0pt}>{\raggedright}p{\dimexpr 0.75in+0\tabcolsep+0\arrayrulewidth}}{\fontsize{11}{11}\selectfont{\textcolor[HTML]{000000}{\global\setmainfont{Arial}{Intercept}}}} & \multicolumn{1}{!{\color[HTML]{000000}\vrule width 0pt}>{\raggedleft}p{\dimexpr 0.75in+0\tabcolsep+0\arrayrulewidth}}{\fontsize{11}{11}\selectfont{\textcolor[HTML]{000000}{\global\setmainfont{Arial}{0.4116}}}} & \multicolumn{1}{!{\color[HTML]{000000}\vrule width 0pt}>{\raggedleft}p{\dimexpr 0.75in+0\tabcolsep+0\arrayrulewidth}}{\fontsize{11}{11}\selectfont{\textcolor[HTML]{000000}{\global\setmainfont{Arial}{1}}}} & \multicolumn{1}{!{\color[HTML]{000000}\vrule width 0pt}>{\raggedleft}p{\dimexpr 0.75in+0\tabcolsep+0\arrayrulewidth}}{\fontsize{11}{11}\selectfont{\textcolor[HTML]{000000}{\global\setmainfont{Arial}{24,715.22}}}} & \multicolumn{1}{!{\color[HTML]{000000}\vrule width 0pt}>{\raggedleft}p{\dimexpr 0.75in+0\tabcolsep+0\arrayrulewidth}!{\color[HTML]{000000}\vrule width 0pt}}{\fontsize{11}{11}\selectfont{\textcolor[HTML]{000000}{\global\setmainfont{Arial}{0.0000}}}} \\





\multicolumn{1}{!{\color[HTML]{000000}\vrule width 0pt}>{\raggedright}p{\dimexpr 0.75in+0\tabcolsep+0\arrayrulewidth}}{\fontsize{11}{11}\selectfont{\textcolor[HTML]{000000}{\global\setmainfont{Arial}{LL}}}} & \multicolumn{1}{!{\color[HTML]{000000}\vrule width 0pt}>{\raggedright}p{\dimexpr 0.75in+0\tabcolsep+0\arrayrulewidth}}{\fontsize{11}{11}\selectfont{\textcolor[HTML]{000000}{\global\setmainfont{Arial}{\%\ sand}}}} & \multicolumn{1}{!{\color[HTML]{000000}\vrule width 0pt}>{\raggedleft}p{\dimexpr 0.75in+0\tabcolsep+0\arrayrulewidth}}{\fontsize{11}{11}\selectfont{\textcolor[HTML]{000000}{\global\setmainfont{Arial}{0.0659}}}} & \multicolumn{1}{!{\color[HTML]{000000}\vrule width 0pt}>{\raggedleft}p{\dimexpr 0.75in+0\tabcolsep+0\arrayrulewidth}}{\fontsize{11}{11}\selectfont{\textcolor[HTML]{000000}{\global\setmainfont{Arial}{1}}}} & \multicolumn{1}{!{\color[HTML]{000000}\vrule width 0pt}>{\raggedleft}p{\dimexpr 0.75in+0\tabcolsep+0\arrayrulewidth}}{\fontsize{11}{11}\selectfont{\textcolor[HTML]{000000}{\global\setmainfont{Arial}{3,954.04}}}} & \multicolumn{1}{!{\color[HTML]{000000}\vrule width 0pt}>{\raggedleft}p{\dimexpr 0.75in+0\tabcolsep+0\arrayrulewidth}!{\color[HTML]{000000}\vrule width 0pt}}{\fontsize{11}{11}\selectfont{\textcolor[HTML]{000000}{\global\setmainfont{Arial}{0.0000}}}} \\





\multicolumn{1}{!{\color[HTML]{000000}\vrule width 0pt}>{\raggedright}p{\dimexpr 0.75in+0\tabcolsep+0\arrayrulewidth}}{\fontsize{11}{11}\selectfont{\textcolor[HTML]{000000}{\global\setmainfont{Arial}{LL}}}} & \multicolumn{1}{!{\color[HTML]{000000}\vrule width 0pt}>{\raggedright}p{\dimexpr 0.75in+0\tabcolsep+0\arrayrulewidth}}{\fontsize{11}{11}\selectfont{\textcolor[HTML]{000000}{\global\setmainfont{Arial}{Uniformity}}}} & \multicolumn{1}{!{\color[HTML]{000000}\vrule width 0pt}>{\raggedleft}p{\dimexpr 0.75in+0\tabcolsep+0\arrayrulewidth}}{\fontsize{11}{11}\selectfont{\textcolor[HTML]{000000}{\global\setmainfont{Arial}{0.0001}}}} & \multicolumn{1}{!{\color[HTML]{000000}\vrule width 0pt}>{\raggedleft}p{\dimexpr 0.75in+0\tabcolsep+0\arrayrulewidth}}{\fontsize{11}{11}\selectfont{\textcolor[HTML]{000000}{\global\setmainfont{Arial}{1}}}} & \multicolumn{1}{!{\color[HTML]{000000}\vrule width 0pt}>{\raggedleft}p{\dimexpr 0.75in+0\tabcolsep+0\arrayrulewidth}}{\fontsize{11}{11}\selectfont{\textcolor[HTML]{000000}{\global\setmainfont{Arial}{6.00}}}} & \multicolumn{1}{!{\color[HTML]{000000}\vrule width 0pt}>{\raggedleft}p{\dimexpr 0.75in+0\tabcolsep+0\arrayrulewidth}!{\color[HTML]{000000}\vrule width 0pt}}{\fontsize{11}{11}\selectfont{\textcolor[HTML]{000000}{\global\setmainfont{Arial}{0.0193}}}} \\





\multicolumn{1}{!{\color[HTML]{000000}\vrule width 0pt}>{\raggedright}p{\dimexpr 0.75in+0\tabcolsep+0\arrayrulewidth}}{\fontsize{11}{11}\selectfont{\textcolor[HTML]{000000}{\global\setmainfont{Arial}{LL}}}} & \multicolumn{1}{!{\color[HTML]{000000}\vrule width 0pt}>{\raggedright}p{\dimexpr 0.75in+0\tabcolsep+0\arrayrulewidth}}{\fontsize{11}{11}\selectfont{\textcolor[HTML]{000000}{\global\setmainfont{Arial}{\%\ sand\ x\ Uniformity}}}} & \multicolumn{1}{!{\color[HTML]{000000}\vrule width 0pt}>{\raggedleft}p{\dimexpr 0.75in+0\tabcolsep+0\arrayrulewidth}}{\fontsize{11}{11}\selectfont{\textcolor[HTML]{000000}{\global\setmainfont{Arial}{0.0000}}}} & \multicolumn{1}{!{\color[HTML]{000000}\vrule width 0pt}>{\raggedleft}p{\dimexpr 0.75in+0\tabcolsep+0\arrayrulewidth}}{\fontsize{11}{11}\selectfont{\textcolor[HTML]{000000}{\global\setmainfont{Arial}{1}}}} & \multicolumn{1}{!{\color[HTML]{000000}\vrule width 0pt}>{\raggedleft}p{\dimexpr 0.75in+0\tabcolsep+0\arrayrulewidth}}{\fontsize{11}{11}\selectfont{\textcolor[HTML]{000000}{\global\setmainfont{Arial}{0.06}}}} & \multicolumn{1}{!{\color[HTML]{000000}\vrule width 0pt}>{\raggedleft}p{\dimexpr 0.75in+0\tabcolsep+0\arrayrulewidth}!{\color[HTML]{000000}\vrule width 0pt}}{\fontsize{11}{11}\selectfont{\textcolor[HTML]{000000}{\global\setmainfont{Arial}{0.8065}}}} \\





\multicolumn{1}{!{\color[HTML]{000000}\vrule width 0pt}>{\raggedright}p{\dimexpr 0.75in+0\tabcolsep+0\arrayrulewidth}}{\fontsize{11}{11}\selectfont{\textcolor[HTML]{000000}{\global\setmainfont{Arial}{LL}}}} & \multicolumn{1}{!{\color[HTML]{000000}\vrule width 0pt}>{\raggedright}p{\dimexpr 0.75in+0\tabcolsep+0\arrayrulewidth}}{\fontsize{11}{11}\selectfont{\textcolor[HTML]{000000}{\global\setmainfont{Arial}{Residuals}}}} & \multicolumn{1}{!{\color[HTML]{000000}\vrule width 0pt}>{\raggedleft}p{\dimexpr 0.75in+0\tabcolsep+0\arrayrulewidth}}{\fontsize{11}{11}\selectfont{\textcolor[HTML]{000000}{\global\setmainfont{Arial}{0.0006}}}} & \multicolumn{1}{!{\color[HTML]{000000}\vrule width 0pt}>{\raggedleft}p{\dimexpr 0.75in+0\tabcolsep+0\arrayrulewidth}}{\fontsize{11}{11}\selectfont{\textcolor[HTML]{000000}{\global\setmainfont{Arial}{36}}}} & \multicolumn{1}{!{\color[HTML]{000000}\vrule width 0pt}>{\raggedleft}p{\dimexpr 0.75in+0\tabcolsep+0\arrayrulewidth}}{\fontsize{11}{11}\selectfont{\textcolor[HTML]{000000}{\global\setmainfont{Arial}{}}}} & \multicolumn{1}{!{\color[HTML]{000000}\vrule width 0pt}>{\raggedleft}p{\dimexpr 0.75in+0\tabcolsep+0\arrayrulewidth}!{\color[HTML]{000000}\vrule width 0pt}}{\fontsize{11}{11}\selectfont{\textcolor[HTML]{000000}{\global\setmainfont{Arial}{}}}} \\





\multicolumn{1}{!{\color[HTML]{000000}\vrule width 0pt}>{\raggedright}p{\dimexpr 0.75in+0\tabcolsep+0\arrayrulewidth}}{\fontsize{11}{11}\selectfont{\textcolor[HTML]{000000}{\global\setmainfont{Arial}{PL}}}} & \multicolumn{1}{!{\color[HTML]{000000}\vrule width 0pt}>{\raggedright}p{\dimexpr 0.75in+0\tabcolsep+0\arrayrulewidth}}{\fontsize{11}{11}\selectfont{\textcolor[HTML]{000000}{\global\setmainfont{Arial}{Intercept}}}} & \multicolumn{1}{!{\color[HTML]{000000}\vrule width 0pt}>{\raggedleft}p{\dimexpr 0.75in+0\tabcolsep+0\arrayrulewidth}}{\fontsize{11}{11}\selectfont{\textcolor[HTML]{000000}{\global\setmainfont{Arial}{0.1315}}}} & \multicolumn{1}{!{\color[HTML]{000000}\vrule width 0pt}>{\raggedleft}p{\dimexpr 0.75in+0\tabcolsep+0\arrayrulewidth}}{\fontsize{11}{11}\selectfont{\textcolor[HTML]{000000}{\global\setmainfont{Arial}{1}}}} & \multicolumn{1}{!{\color[HTML]{000000}\vrule width 0pt}>{\raggedleft}p{\dimexpr 0.75in+0\tabcolsep+0\arrayrulewidth}}{\fontsize{11}{11}\selectfont{\textcolor[HTML]{000000}{\global\setmainfont{Arial}{8,914.39}}}} & \multicolumn{1}{!{\color[HTML]{000000}\vrule width 0pt}>{\raggedleft}p{\dimexpr 0.75in+0\tabcolsep+0\arrayrulewidth}!{\color[HTML]{000000}\vrule width 0pt}}{\fontsize{11}{11}\selectfont{\textcolor[HTML]{000000}{\global\setmainfont{Arial}{0.0000}}}} \\





\multicolumn{1}{!{\color[HTML]{000000}\vrule width 0pt}>{\raggedright}p{\dimexpr 0.75in+0\tabcolsep+0\arrayrulewidth}}{\fontsize{11}{11}\selectfont{\textcolor[HTML]{000000}{\global\setmainfont{Arial}{PL}}}} & \multicolumn{1}{!{\color[HTML]{000000}\vrule width 0pt}>{\raggedright}p{\dimexpr 0.75in+0\tabcolsep+0\arrayrulewidth}}{\fontsize{11}{11}\selectfont{\textcolor[HTML]{000000}{\global\setmainfont{Arial}{\%\ sand}}}} & \multicolumn{1}{!{\color[HTML]{000000}\vrule width 0pt}>{\raggedleft}p{\dimexpr 0.75in+0\tabcolsep+0\arrayrulewidth}}{\fontsize{11}{11}\selectfont{\textcolor[HTML]{000000}{\global\setmainfont{Arial}{0.0137}}}} & \multicolumn{1}{!{\color[HTML]{000000}\vrule width 0pt}>{\raggedleft}p{\dimexpr 0.75in+0\tabcolsep+0\arrayrulewidth}}{\fontsize{11}{11}\selectfont{\textcolor[HTML]{000000}{\global\setmainfont{Arial}{1}}}} & \multicolumn{1}{!{\color[HTML]{000000}\vrule width 0pt}>{\raggedleft}p{\dimexpr 0.75in+0\tabcolsep+0\arrayrulewidth}}{\fontsize{11}{11}\selectfont{\textcolor[HTML]{000000}{\global\setmainfont{Arial}{928.96}}}} & \multicolumn{1}{!{\color[HTML]{000000}\vrule width 0pt}>{\raggedleft}p{\dimexpr 0.75in+0\tabcolsep+0\arrayrulewidth}!{\color[HTML]{000000}\vrule width 0pt}}{\fontsize{11}{11}\selectfont{\textcolor[HTML]{000000}{\global\setmainfont{Arial}{0.0000}}}} \\





\multicolumn{1}{!{\color[HTML]{000000}\vrule width 0pt}>{\raggedright}p{\dimexpr 0.75in+0\tabcolsep+0\arrayrulewidth}}{\fontsize{11}{11}\selectfont{\textcolor[HTML]{000000}{\global\setmainfont{Arial}{PL}}}} & \multicolumn{1}{!{\color[HTML]{000000}\vrule width 0pt}>{\raggedright}p{\dimexpr 0.75in+0\tabcolsep+0\arrayrulewidth}}{\fontsize{11}{11}\selectfont{\textcolor[HTML]{000000}{\global\setmainfont{Arial}{Uniformity}}}} & \multicolumn{1}{!{\color[HTML]{000000}\vrule width 0pt}>{\raggedleft}p{\dimexpr 0.75in+0\tabcolsep+0\arrayrulewidth}}{\fontsize{11}{11}\selectfont{\textcolor[HTML]{000000}{\global\setmainfont{Arial}{0.0000}}}} & \multicolumn{1}{!{\color[HTML]{000000}\vrule width 0pt}>{\raggedleft}p{\dimexpr 0.75in+0\tabcolsep+0\arrayrulewidth}}{\fontsize{11}{11}\selectfont{\textcolor[HTML]{000000}{\global\setmainfont{Arial}{1}}}} & \multicolumn{1}{!{\color[HTML]{000000}\vrule width 0pt}>{\raggedleft}p{\dimexpr 0.75in+0\tabcolsep+0\arrayrulewidth}}{\fontsize{11}{11}\selectfont{\textcolor[HTML]{000000}{\global\setmainfont{Arial}{1.32}}}} & \multicolumn{1}{!{\color[HTML]{000000}\vrule width 0pt}>{\raggedleft}p{\dimexpr 0.75in+0\tabcolsep+0\arrayrulewidth}!{\color[HTML]{000000}\vrule width 0pt}}{\fontsize{11}{11}\selectfont{\textcolor[HTML]{000000}{\global\setmainfont{Arial}{0.2595}}}} \\





\multicolumn{1}{!{\color[HTML]{000000}\vrule width 0pt}>{\raggedright}p{\dimexpr 0.75in+0\tabcolsep+0\arrayrulewidth}}{\fontsize{11}{11}\selectfont{\textcolor[HTML]{000000}{\global\setmainfont{Arial}{PL}}}} & \multicolumn{1}{!{\color[HTML]{000000}\vrule width 0pt}>{\raggedright}p{\dimexpr 0.75in+0\tabcolsep+0\arrayrulewidth}}{\fontsize{11}{11}\selectfont{\textcolor[HTML]{000000}{\global\setmainfont{Arial}{\%\ sand\ x\ Uniformity}}}} & \multicolumn{1}{!{\color[HTML]{000000}\vrule width 0pt}>{\raggedleft}p{\dimexpr 0.75in+0\tabcolsep+0\arrayrulewidth}}{\fontsize{11}{11}\selectfont{\textcolor[HTML]{000000}{\global\setmainfont{Arial}{0.0000}}}} & \multicolumn{1}{!{\color[HTML]{000000}\vrule width 0pt}>{\raggedleft}p{\dimexpr 0.75in+0\tabcolsep+0\arrayrulewidth}}{\fontsize{11}{11}\selectfont{\textcolor[HTML]{000000}{\global\setmainfont{Arial}{1}}}} & \multicolumn{1}{!{\color[HTML]{000000}\vrule width 0pt}>{\raggedleft}p{\dimexpr 0.75in+0\tabcolsep+0\arrayrulewidth}}{\fontsize{11}{11}\selectfont{\textcolor[HTML]{000000}{\global\setmainfont{Arial}{1.64}}}} & \multicolumn{1}{!{\color[HTML]{000000}\vrule width 0pt}>{\raggedleft}p{\dimexpr 0.75in+0\tabcolsep+0\arrayrulewidth}!{\color[HTML]{000000}\vrule width 0pt}}{\fontsize{11}{11}\selectfont{\textcolor[HTML]{000000}{\global\setmainfont{Arial}{0.2101}}}} \\





\multicolumn{1}{!{\color[HTML]{000000}\vrule width 0pt}>{\raggedright}p{\dimexpr 0.75in+0\tabcolsep+0\arrayrulewidth}}{\fontsize{11}{11}\selectfont{\textcolor[HTML]{000000}{\global\setmainfont{Arial}{PL}}}} & \multicolumn{1}{!{\color[HTML]{000000}\vrule width 0pt}>{\raggedright}p{\dimexpr 0.75in+0\tabcolsep+0\arrayrulewidth}}{\fontsize{11}{11}\selectfont{\textcolor[HTML]{000000}{\global\setmainfont{Arial}{Residuals}}}} & \multicolumn{1}{!{\color[HTML]{000000}\vrule width 0pt}>{\raggedleft}p{\dimexpr 0.75in+0\tabcolsep+0\arrayrulewidth}}{\fontsize{11}{11}\selectfont{\textcolor[HTML]{000000}{\global\setmainfont{Arial}{0.0004}}}} & \multicolumn{1}{!{\color[HTML]{000000}\vrule width 0pt}>{\raggedleft}p{\dimexpr 0.75in+0\tabcolsep+0\arrayrulewidth}}{\fontsize{11}{11}\selectfont{\textcolor[HTML]{000000}{\global\setmainfont{Arial}{30}}}} & \multicolumn{1}{!{\color[HTML]{000000}\vrule width 0pt}>{\raggedleft}p{\dimexpr 0.75in+0\tabcolsep+0\arrayrulewidth}}{\fontsize{11}{11}\selectfont{\textcolor[HTML]{000000}{\global\setmainfont{Arial}{}}}} & \multicolumn{1}{!{\color[HTML]{000000}\vrule width 0pt}>{\raggedleft}p{\dimexpr 0.75in+0\tabcolsep+0\arrayrulewidth}!{\color[HTML]{000000}\vrule width 0pt}}{\fontsize{11}{11}\selectfont{\textcolor[HTML]{000000}{\global\setmainfont{Arial}{}}}} \\

\hhline{>{\arrayrulecolor[HTML]{666666}\global\arrayrulewidth=2pt}->{\arrayrulecolor[HTML]{666666}\global\arrayrulewidth=2pt}->{\arrayrulecolor[HTML]{666666}\global\arrayrulewidth=2pt}->{\arrayrulecolor[HTML]{666666}\global\arrayrulewidth=2pt}->{\arrayrulecolor[HTML]{666666}\global\arrayrulewidth=2pt}->{\arrayrulecolor[HTML]{666666}\global\arrayrulewidth=2pt}-}



\end{longtable}

\hypertarget{uniformity-effect-on-apparent-threshold-fines-content}{%
\subsubsection{Uniformity effect on apparent threshold fines content}\label{uniformity-effect-on-apparent-threshold-fines-content}}

Another feature of Experiment 2 was that the mixtures containing different C\textsubscript{u} sands became nonplastic at different sand contents.
Low-C\textsubscript{u} sand mixes containing 70, 72.5, and 75\% sand could not be rolled into threads, but high-C\textsubscript{u} mixes at these sand contents still had measurable plastic limits.
These observations suggest that the threshold fines content is affected by sand uniformity and merits further study.
Calculations of the threshold fines content rarely coincide with actual soil behavior observed in experiments (\protect\hyperlink{ref-Zuo2015}{Zuo and Baudet, 2015}).
However, little attention has been paid to soil mixtures containing plastic fines as opposed to nonplastic silt or two granular materials.
While not replicated or tested across multiple clay types, further research may demonstrate that TFC can be accurately computed from the minimum void ratio of a sand without performing time-consuming experiments.

\hypertarget{conclusions}{%
\section{Conclusions}\label{conclusions}}

Standard Atterberg limit test protocols limit the particle size range to \textless425 μm.
Parts I and II of this research have each demonstrated that the tests can be accurately performed when the mixtures contain particles 2000-425 μm, even when the mixtures contain up to \textasciitilde70\% added coarse particles.
When coarse sand is used, there is a strong linear relationship between sand content and LL or PL up to ≥60\% sand.

Sand angularity had no effect on LL or PL in Experiment 1.
Sand-size uniformity affected LL but not PL in Experiment 2. High-Cu mixtures required more water to wet them to the LL than low-C¬u mixtures.
This trend could be due to the higher specific surface area of the high-Cu sand.

The high-Cu sand mixtures in Experiment 2 also became nonplastic at relatively lower sand content than low-Cu mixtures.
This implies a predictive model might be developed from Cu and/or void ratio to compute the threshold fines content marking a change from plastic to nonplastic behavior.

Collectively, the results of Parts I and II demonstrate that Atterberg limits can be performed with significant amounts of particles coarser than the current limit specified in ASTM D4318.
Sand angularity and sand-size uniformity have less influence on Atterberg limits of soil mixtures than sand particle size and total sand content.

\hypertarget{data-availability}{%
\section{Data availability}\label{data-availability}}

Data are currently being uploaded to Dryad.

\hypertarget{references}{%
\section*{References}\label{references}}
\addcontentsline{toc}{section}{References}

\hypertarget{refs}{}
\begin{CSLReferences}{1}{0}
\leavevmode\vadjust pre{\hypertarget{ref-Adams1994}{}}%
Adams, W.A., and R.J. Gibbs. 1994. Natural {Turf} for {Sport} and {Amenity}. 1st ed. {CABI International}.

\leavevmode\vadjust pre{\hypertarget{ref-ASTMF1815-11}{}}%
ASTM International. 2011. F1815-11 {Standard Test Methods} for {Saturated Hydraulic Conductivity} , {Water Retention} , {Porosity} , and {Bulk Density} of {Athletic Field Rootzones}. : 1--6. doi: \href{https://doi.org/10.1520/F1815-11.2}{10.1520/F1815-11.2}.

\leavevmode\vadjust pre{\hypertarget{ref-ASTMInternational2016}{}}%
ASTM International. 2016. \href{https://doi.org/10.1520/D4253-16}{4253-16: {Test Methods} for {Maximum Index Density} and {Unit Weight} of {Soils Using} a {Vibratory Table}}. {ASTM International}.

\leavevmode\vadjust pre{\hypertarget{ref-ASTMInternational2017}{}}%
ASTM International. 2017. \href{https://doi.org/10.1520/D6913_D6913M-17}{Test {Methods} for {Particle-Size Distribution} ({Gradation}) of {Soils Using Sieve Analysis}}. {ASTM International}.

\leavevmode\vadjust pre{\hypertarget{ref-ASTMD43182018}{}}%
ASTM International. 2018. D4318-17, {Standard Test Methods} for {Liquid Limit}, {Plastic Limit}, and {Plasticity Index} of {Soils}. : 20. doi: \href{https://doi.org/10.1520/D4318-17E01.}{10.1520/D4318-17E01.}

\leavevmode\vadjust pre{\hypertarget{ref-Atterberg1911}{}}%
Atterberg, A. 1911. Die {Plastizität} der {Tone}. Intern mitt. boden: 4--37.

\leavevmode\vadjust pre{\hypertarget{ref-Baker2006a}{}}%
Baker, B., Stephen W. (Sports Turf Research Institute). 2006. Rootzones, {Sands} and {Top Dressing Materials} for {Sports Turf}. {Sports Turf Research Institute}, {West Yorkshire, England}.

\leavevmode\vadjust pre{\hypertarget{ref-Barnes2013}{}}%
Barnes, G.E. 2013. The {Plastic Limit} and {Workability} of {Soils}. \url{https://www.escholar.manchester.ac.uk/api/datastream?publicationPid=uk-ac-man-scw:212752\&datastreamId=FULL-TEXT.PDF}.

\leavevmode\vadjust pre{\hypertarget{ref-Dumbleton1966b}{}}%
Dumbleton, M.J., and G. West. 1966. The influence of the coarse fraction on the plastic properties of clay soils- {RRL Report No}. 36. {Road Research Laboratory}, {Crowthorne, Berkshire}.

\leavevmode\vadjust pre{\hypertarget{ref-Gee2002}{}}%
Gee, G.W., and D. Or. 2002. Particle-size analysis. In: Dane, J.H. and Topp, C.G., editors, Methods of {Soil Analysis}: {Part} 4 {Physical Methods}. {Soil Science Society of America}, {Madison, WI}. p. 255--293

\leavevmode\vadjust pre{\hypertarget{ref-GNU2020}{}}%
GNU. 2020. \href{https://www.gnu.org/software/make/}{{GNU Make}}. {The GNU Project}.

\leavevmode\vadjust pre{\hypertarget{ref-Holtz2010}{}}%
Holtz, R.D., W.D. Kovacs, and T.C. Sheahan. 2010. An {Introduction} to {Geotechnical Engineering}. {Pearson}, {New York, NY}.

\leavevmode\vadjust pre{\hypertarget{ref-Krumbein1941}{}}%
Krumbein, W.C. 1941. Measurement and {Geological Significance} of {Shape} and {Roundness} of {Sedimentary Particles}. SEPM JSR Vol. 11. doi: \href{https://doi.org/10.1306/D42690F3-2B26-11D7-8648000102C1865D}{10.1306/D42690F3-2B26-11D7-8648000102C1865D}.

\leavevmode\vadjust pre{\hypertarget{ref-Lade1998}{}}%
Lade, P.V., C.D. Liggio Jr., and J.A. Yamamuro. 1998. Effects of {Particle Shapes} and {Sizes} on the {Minimum Void Ratios} of {Sand}. Geotechnical Testing Journal 21(4): 336--347.

\leavevmode\vadjust pre{\hypertarget{ref-Miller2011}{}}%
Miller, N.A., and J.J. Henderson. 2011. Correlating {Particle Shape Parameters} to {Bulk Properties} and {Load Stress} at {Two Water Contents}. Agron. J. 103(5): 1514--1523. doi: \href{https://doi.org/10.2134/agronj2010.0235}{10.2134/agronj2010.0235}.

\leavevmode\vadjust pre{\hypertarget{ref-Mitchell1993}{}}%
Mitchell, J.K., and K. Soga. 1993. Fundamentals of {Soil Behavior}. 3rd ed. {Wiley}.

\leavevmode\vadjust pre{\hypertarget{ref-Mittal2004}{}}%
Mittal, B., H. Yi, V.M. Puri, A.S. McNitt, and C.F. Mancino. 2004. Bulk {Mechanical Behavior} of {Rootzone Sand Mixtures} as {Influenced} by {Particle Shape}, {Moisture} and {Peat}. Part. Part. Syst. Charact. 21(4): 303--309. doi: \href{https://doi.org/10.1002/ppsc.200400934}{10.1002/ppsc.200400934}.

\leavevmode\vadjust pre{\hypertarget{ref-R-Core-Team2022}{}}%
R-Core-Team. 2022. \href{https://www.r-project.org/}{R: {A} language and environment for statistical computing}. {R Foundation for Statistical Computing}, {Vienna, Austria}.

\leavevmode\vadjust pre{\hypertarget{ref-Sibley2020}{}}%
Sibley, E.L.D., and C.P. Polito. 2020. \href{https://geovirtual2020.ca/wp-content/files/406.pdf}{Insights on {Threshold Fines Content}}. Geovirtual 2020 {Resilience} and {Innovation}

\leavevmode\vadjust pre{\hypertarget{ref-Sivapullaiah1985}{}}%
Sivapullaiah, P.V., and A. Sridharan. 1985. Liquid {Limit} of {Soil Mixtures}. Geotechnical Testing Journal 8(3): 111--116. doi: \href{https://doi.org/10.1520/gtj10521j}{10.1520/gtj10521j}.

\leavevmode\vadjust pre{\hypertarget{ref-Suhr2020}{}}%
Suhr, B., W.A. Skipper, R. Lewis, and K. Six. 2020. Shape analysis of railway ballast stones: Curvature-based calculation of particle angularity. Sci Rep 10(1): 6045. doi: \href{https://doi.org/10.1038/s41598-020-62827-w}{10.1038/s41598-020-62827-w}.

\leavevmode\vadjust pre{\hypertarget{ref-Wadell1932}{}}%
Wadell, H. 1932. Volume, {Shape}, and {Roundness} of {Rock Particles}. The Journal of Geology 40(5): 443--451. \url{https://www.jstor.org/stable/30058012}.

\leavevmode\vadjust pre{\hypertarget{ref-Zuo2015}{}}%
Zuo, L., and B.A. Baudet. 2015. Determination of the transitional fines content of sand-non plastic fines mixtures. Soils and Foundations 55(1): 213--219. doi: \href{https://doi.org/10.1016/j.sandf.2014.12.017}{10.1016/j.sandf.2014.12.017}.

\end{CSLReferences}

\end{document}
